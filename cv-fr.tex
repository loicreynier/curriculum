% LTeX: language=fr

\documentclass[alternative]{yaac-another-awesome-cv}
\name{Loïc}{Reynier}
\tagline{Ingénieur en Mécanique | Doctorant en Mécanique des Fluides}
\photo{3.5cm}{assets/Loic_Reynier_EFMC14}
\socialinfo{
  \linkedin{loicreynier}
  \github{loicreynier}
  \email{loic@loicreynier.fr}}

\hypersetup{
  pdfauthor={Loïc Reynier},
  pdftitle={CV Loïc Reynier},
}

\addbibresource{bib/references.bib}
\begin{document}

\makecvheader
\makecvfooter
  {\textsc{\today}} % \selectlanguage{english}\today
  {\textsc{Loïc Reynier - CV}}
  {\thepage}

\sectionTitle{Formations}{\faGraduationCap}
\begin{scholarship}
  \scholarshipentry{2020-2023}
    {Doctorat en Mécanique des Fluides au LMFA (Université Lyon1)}
  \scholarshipentry{2017-2020}
    {Diplôme d'Ingénieur en Mécanique de Polytech Lyon (Université Lyon1)}
  \scholarshipentry{2017-2020}
    {Master Mécanique des Fluides et Énergétique de l'Université Lyon1 (double diplôme)}
  \scholarshipentry{2015-2017}
    {Classes préparatoires aux grandes écoles à Polytech Marseille (Université Aix-Marseille)}
\end{scholarship}

\sectionTitle{Expériences de Recherche}{\faFlask}
\begin{experiences}
  \experience
    {Aujourd'hui}
    {Thèse en mécanique des fluides}{LMFA (Université Lyon1)}{Lyon, France}
    {Octobre 2020}
    {
      \textit{Transition et turbulence des écoulements incompressibles à densité variable}
      \begin{itemize}
        \item Développement d'un forçage pour la turbulence isotrope à densité variable
        \item Simulation DNS et analyse statistique de la turbulence isotrope
        \item Simulation DNS et analyse statistique de la transition d'un jet temporel à densité variable\\
              vers la turbulence isotrope
      \end{itemize}
    }
    {
      densité variable,
      NS incompressible,
      turbulence,
      DNS,
      méthodes spectrales,
      schémas implicites,
      préconditionnement
    }
  \emptySeparator
  \experience
    {Septembre 2020}
    {Stage R\&D de fin d'études}{LMFA (Université Lyon1)}{Lyon, France}
    {Avril 2020}
    {
      \textit{Simulation haute performance des écoulements inhomogènes}
      \begin{itemize}
        \item Développement de schémas d'ordres élevés semi-implicites pour NS incompressible
        \item Validation des schémas sur des simulations hautes résolutions
        \item Création d'un module d'analyse statistique des écoulements turbulents
      \end{itemize}
    }
    {
      densité variable,
      NS incompressible,
      turbulence,
      DNS,
      méthodes spectrales,
      schémas implicites,
      préconditionnement
    }
  \emptySeparator
  \experience
    {Août 2019}
    {Stage R\&D}{Louisiana State University}{Bâton Rouge, USA}
    {Mars 2019}
    {
      \textit{Simulation de l'atomisation par forces centrifuges de poudre métallique pour fabrication additive}
      \begin{itemize}
        \item Développement de modèles volume-of-fluid et level-set
        \item Simulation de l'atomisation et comparaison des modèles avec les modèles turbulents d'Ansys Fluent
        \item Étude statistique de l'atomisation
      \end{itemize}
    }
    {
      diphasique,
      level-set method,
      vof method,
      volumes finis,
      Ansys Fluent
    }

\end{experiences}

\sectionTitle{Publications}{\faBook}
\begin{publications}
  \nocite{*}
  \printbibliography[heading=none]
\end{publications}

\sectionTitle{Compétences}{\faTasks}
\begin{keywords}
  \keywordsentry{Simulation Numérique}{
    \textbf{Méthodes spectrales},
    Méthodes de Krylov,
    Différences/Volumes/Éléments Fini(e)s,
    Schémas d'ordres élevés,
    Schémas implicites,
    \textbf{DNS}}
  \keywordsentry{Programmation}{
    C/C++,
    \textbf{Fortran},
    \textbf{Python},
    Rust,
    \textbf{MPI},
    OpenMP,
    \textbf{CUDA},
    Dash,
    \textbf{FFTW},
    HDF5}
  \keywordsentry{Outils de développement}{
    \textbf{Linux},
    \textbf{Nix},
    SLURM,
    \textbf{Git},
    Visual Studio,
    VS Code,
    \textbf{(Neo)vim}}
  \keywordsentry{Langues}{
    Français (langue maternelle),
    Anglais (C1)}
\end{keywords}

\sectionTitle{Références}{\faQuoteLeft}
\begin{referees}
  \referee
    {Bastien Di Pierro}
    {Maître de Conférences}
    {Université Lyon1, LMFA}
    {bastien.di-pierro@univ-lyon1.fr}
    {r4 72 43 14 44}
  \hfill
  \referee
    {Frédéric Alizard}
    {Maître de Conférences}
    {Université Lyon1, LMFA}
    {frederic.alizard@univ-lyon1.fr}
    {04 72 18 61 54}
\end{referees}

\end{document}
